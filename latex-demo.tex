%% LaTeX的注释就是‘%’
%  采用的环境是MikiTeX+xelatex,编辑器是vscode,Microsoft一个开源编辑器,功能比较强大
%  LaTeX 主要用来排版论文、书籍、海报还有信件等等
%  LaTeX文档的结构
\documentclass[utf8,a4paper,12pt]{ctexart}
\usepackage[left=2.50cm,right=2.50cm,top=2cm,bottom=2cm]{geometry}
%% 美国数学学会包
\usepackage{amssymb,amsfonts,amsmath,amsthm}
\usepackage{graphics}
\usepackage{tikz}
\title{\LaTeX 学习}
\author{李宁}
\date{}
\begin{document}
    \maketitle
    \tableofcontents
    \begin{abstract}
        这是一个摘要,这是一个摘要,这是一个摘要,这是一个摘要,这是一个摘要,这是一个摘要,这是一个摘要,这是一个摘要,这是一个摘要,这是一个摘要。
    \end{abstract}
    \section{引言}
    \LaTeX 是一个比较好用的排版平台,但是入门比较困难,相比较word而言,但是排版比较大的文件,不会死机,并且可以多人分工合作,采用一样的模板,主要用在数学、物理教材可以用。作图可以用figure这个包,把文件路径引进来就可以了。

    \LaTeX 的入门相对困难一些,这个系统平台的安装包也比较大,一般接近3G的大小。百度搜索CTex,中科院的有个博士论文包,可以参考学习。

    可以排版优美的数学公式。
    \section{数学公式的排版}
    数学公式的排版,我们要掌握latex语言,比较繁琐,不过可以mathtype导出TeX代码,也可以手写图片,采用mathpix snipping tools,识别率到90\%。

    数学公式分为二种,一种叫做行内公式,$v=v_0+at$,还有一种叫做块状公式,
    \[
        x=v_0 t+\frac{1}{2} at^2   
    \]
    带编号的公式
    \begin{equation}%%% equation环境,可以自动编号
        \nabla \cdot E =\frac{\rho_0}{\varepsilon_0}
    \end{equation}
    多个公式采用一个编号
    \begin{equation}
        \begin{split}
            \nabla \cdot E &=\frac{\rho_0}{\varepsilon_0}\\
            \nabla \times E&=0  ~~~~    \text{麦克斯韦方程组}
        \end{split}
    \end{equation}
    测试一下
    \section{图片的制作}
    \subsection{figure包下的图片引入}
    \begin{figure}
        \centering
        \includegraphics[width=0.3\textwidth]{0013451.png}
        \caption{摩擦角}
        \label{fraction}
    \end{figure}
    根据图(\ref{fraction}),图片引入方式,这是位图,也可以pdf,也可以是eps。
    \subsection{tikz图片的生成}
    \begin{tikzpicture}
        \draw (0,0) circle (3cm);
        \draw (6,0) rectangle (12,2);
    \end{tikzpicture}
    推荐学习的网站LaTeX Studio,这个网站有很多教程,也有很多模板,tikz网站也有很多模板,自己尝试着去改。
\end{document}
